\chapter{Experiments}\label{sec:experiments}

In this section, the performed experiments are presented, and results of the
experiments are shown. There are two experiments: the first experiment set
uses the \mrlco model and the second experiment is with \code{keras-rl} and
OpenAI Gym. The two experiments were performed on the workstation described in
Section \ref{sec:setup}. Both experiments were programmed in Python, and
open-source software projects for the models and environment were used
extensively. Code is open-sourced in a Github
repository\footlink{github.com/yochem/thesis}.

\section{Characterizing Meta Reinforcement Learning}

The objective of the first experiment was to evaluate the adaptiveness and
reduce in retraining time with meta \gls{rl}. The evaluation is done using the
meta \gls{rl} offloading model, called \mrlco, proposed in \citeA{wang2020}.
The data was converted into \DAG data (see \ref{sec:data}). Running the
experiments with the converted data required modifications in the source code
of \gls{mrlco}. The modifications were made in the \code{meta\_trainer.py}
file. There was a list of directories containing 100 graph data files, which
was changed into the directories with the Euro-Argo data. More modifications
have been done to fix bugs, e.g. adding in logic for taking the maximum value
of an array when it is empty.

After the bug fixing, running the model was possible. Each run consisted of
2,000 iterations in batch sizes of 100. After each batch a TensorFlow
checkpoint was saved on disk. Every iteration it appended a row of data to a
comma separated value (csv) file. The following aspects of each iteration were
saved:

\begin{itemize}[noitemsep]
    \item Iteration
    \item Number of trajectories
    \item Average reward
    \item Execution time:
        \begin{itemize}[noitemsep]
            \item Policy execution time
            \item Environment execution time
        \end{itemize}
    \item Return value:
        \begin{itemize}[noitemsep]
            \item Average return value
            \item Minimal return value
            \item Maximal return value
            \item Standard deviation of return value
            \item Average discounted return
        \end{itemize}
    \item Latency:
        \begin{itemize}[noitemsep]
            \item Average latency
            \item Average greedy latency
        \end{itemize}
\end{itemize}

\begin{figure}[htp!]
    \centering
    \includegraphics[width=\textwidth]{./fig/return.pdf}
    \caption{The progress of the return value during training. The upper line
    plots the maximal return value per iteration, the middle line the average
    return value and the lower line plots the minimal return value per
    iteration.}
    \label{fig:return}
\end{figure}



\subsection{Reproducing MRLCO Results}

\begin{figure}[htp!]
    \centering
    \includegraphics[width=.7\textwidth]{./fig/mrlco-reward.pdf}
    \caption{The reward progress of the \mrlco model on the \mrlco data. The
    baseline, without meta-learning, is trained on the first batch of their
    data. After that two models, one with the same data and the other one with
    the second batch of the data, are trained and meta-learn from the
    baseline.}
    \label{mrlco-reward}
\end{figure}


The first experiment was reproducing the results from the paper. This was done
by running the \mrlco model without changes on data provided with the source
code\footnote{\urlmetadata}. The two runs are plotted in Figure
\ref{mrlco-reward}. It is shown that the second runs, that meta-learn,
converge quicker due to meta-learning. The reward in the last iteration of
training was $-5.42$. The average return value at the last iteration was
$-5.60$. A notable difference is that the meta-learned models do not receive
the same reward as the baseline on average. This might be due to an error in
setting the batch sizes, which were smaller than before. This experiment was
performed to compare the results with the results from the paper. The results
in the paper however only show latency, not reward. A comparison can thus not
be made.

\begin{table}[htp!]
    \centering
    \caption{The average policy and environment execution time per run. The
    values are in seconds.}
    \label{table:exectime}
    \begin{tabular}{l|cc}
        & Policy & Environment\\ \hline
        \mrlco Data & 2.477595 & 0.953273\\
        Provided Data \#1 & 2.485432 & 0.953415\\
        Provided Data \#2 & 2.506485 & 0.910931
    \end{tabular}
\end{table}


\begin{figure}[htp!]
    \centering
    \includegraphics[width=.7\textwidth]{./fig/reward.pdf}
    \caption{The progress of average reward during training Euro-Argo log
    data. The data was split in 2 batches of the same size. The blue line is
    the first time learning, so no meta-learning involved. After this, a model
    using the same batch data (orange) and a model using the other batch of
    data (green) were trained.}
    \label{fig:rewards}
\end{figure}


\subsection{MRLCO with Euro-Argo Data}

The second experiment was done with the converted Euro-Argo data. Three
training phases were done. The first training phase was a baseline run,
without any meta-learning up front. It was trained on the first half of the
data: 1500 graph files. This half will from now on be called batch 1, the
other half batch 2. After 1955 iterations the baseline model reached its
optimal reward of -6.7. Now the baseline model is trained, it can be used for
meta-learning. Two meta-learning runs were done: one with the exact same data,
batch 1 and another one with different data from the same distribution, batch
2. These runs reached their optimal reward after respectively 521 and 920
iterations. The real time of these iterations are listed in
\ref{table:max-reward}. In the table it is shown that the training time
speedup on the same data is 105 minutes, from almost three hours to a bit more
than one hour. The training time speedup on new data is 54 minutes. The time
difference between the full training sessions is not significant. In Table
\ref{table:exectime}, the total execution times for policy and environment are
shown. The difference in execution time between the different runs are only a
couple seconds. This means that an iteration is not quicker during
meta-learning. Thus, if meta-learning requires the same number of iterations
to converge as without meta-learning, the retraining time will not be reduced.
The maximal reward on the second data batch is -6.8, a 0.1 difference with the
maximal reward from the base line model. The progress of average reward for
all three training phases are plotted in Figure \ref{fig:rewards}. In this
figure it is clear that the meta-learned runs achieve their optimal policy in
an earlier iteration. Even when the reward at the start is much lower, it
still learns very quick using meta-learning.

The return value is plotted in Figure \ref{fig:return}. The difference in return
values between the runs using Euro-Argo is not significant. There is however a
difference with the \mrlco data, which has a higher minimal return value. No
reason is found for this difference. The maximal and average return value are
around the same value. It can also be seen that when the model meta-learns on
the second batch of the Euro-Argo data, the maximal return value is low at
first but recovers quickly to the same maximal return values as the other
runs. This shows that meta-learning does not affect overall performance of
the model.

\begin{table}[htp!]
    \centering
    \caption{The time it takes to converge to the maximal average reward
    (-6.8). The time is expressed as the number of iterations and the real
    (sometimes called wallclock) time. An experiment of 2000 iterations took
    255 minutes and 10 seconds (15310 seconds). The time to converge thus is
    calculated using the following formula: $\protect it / 2000 \cdot 15310$.}
    \label{table:max-reward}
    \begin{tabular}{c|lrr}
        ID & Batch & Iterations & Time (min)\\ \hline
        1 & 1 & 1345 & 171\\
        2 & 1 & 521 & 66\\
        3 & 2 & 920 & 117\\
    \end{tabular}
\end{table}



\section{Comparing Reinforcement Learning Model in OpenAI Gym}\label{exp:gym}

The second group of experiments were done to evaluate multiple \gls{rl} algorithms
as schedulers and create an environment, in which it is simple to plug in many
\gls{rl} models and approaches to reduce retraining time and test them. This would
allow for comparison between a large amount of schedulers. OpenAI's Gym was
chosen for this because of its popularity and simplicity. As described in
Section \ref{sec:evaluation} and Section \ref{sec:setup} OpenAI is a popular
platform for creating, testing and evaluating reinforcement learning models.
The environment used in this experiment is described in Section
\ref{sec:taillard}. The \jss environment (\code{JSSEnv}) in combination with
the \code{keras-rl} package\footlink{github.com/keras-rl/keras-rl}, would
allow to experiment with many different \gls{rl} algorithms and evaluate their
performance as scheduler. The \code{keras-rl} package currently supports
\acrfull{dqn} \cite{mnih2013}, Double \dqn \cite{hasselt2016}, Deep
Deterministic Policy Gradient \cite{lillicrap2015}, Continuous \dqn
\cite{gu2016}, Cross-Entropy Method \cite{szita2006}, Dueling network \dqn
(Dueling \gls{dqn}) \cite{wang2016}, and Deep SARSA. Simulating the agent in
the \jss environment required very little code. As start make the Gym
environment, after this define a \nn (shown in Figure
\ref{fig:neuralnetwork}), then define an \gls{rl} agent providing a policy and the
\nn and at last fit the agent on the environment.

\begin{figure}[htp!]
    \centering
    \includegraphics[width=\textwidth]{./fig/nn.png}
    \caption{The \glspl{dnn} used in the \gls{rl} models from \code{keras-rl}.}
    \label{fig:neuralnetwork}
\end{figure}


Starting the experiment, a \fifo scheduler in the environment was run. The
makespan after this run was 5724 and the final reward was -5.80, and the
average reward was 0.40. After this general Q-learning and the deep \gls{rl}
algorithms mentioned above were tried. There were no results due to various
errors, which will be discussed in detail in the next section
(§\ref{sec:discussion}).
