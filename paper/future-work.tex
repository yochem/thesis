\chapter{Future Work}\label{sec:future-work}

In this section improvements are suggested for this thesis and for the current
shortcomings in the field. Existing gaps are listed and the importance of
filling in these gaps is explained.


\section{Papers with Code}

Nowadays, many papers include the corresponding
code\footlink{paperswithcode.com}. Although this is generally a good thing
that should be encouraged, it is not always usable yet. All code repositories
that accompanied related work lacked documentation. The papers themself do not
document how code works, and additional documentation is not written either.
Working with the codebase requires reading the written code, which usually
lacks comments and good design patterns, and figuring out how to adjust it to
work with other data, code, or environment. To give an example, the \mrlco
code\footlink{github.com/linkpark/metarl-offloading} only specifies the entry
point of the code and which packages should be installed. The list of packages
is incomplete and it is not specified that an old version of Python should be
installed. Another example from the same code repository is that the command
for generating a specific graph is commented in the first line of the file.
But when running these commands, the jobs lack an \code{expect\_size}
attribute. This attribute is not something that can be done with \code{DAGGEN}
and must be later on added by the authors of the research paper, which should
definitely be documented. Although papers that include code are a good step in
the right direction, it is not very useful yet because of the low-quality code
and documentation. Good practice in future research will be to better document
code and make it available for others to experiment with the code.


\section{Standards in Reinforcement Learning}\label{sec:standards}

In future research, more research is needed to design standards for research
in reinforcement learning. Three standards would significantly improve comparative
research.

The first improvement would be a standard in data formats used in \gls{rl}-based
schedulers. The data formats differ too much to compare \gls{rl}-based schedulers
rightfully using different data formats. The second improvement would be the
exposed API for environments. OpenAIs Gym is a step in the right direction.
The API is clear and straightforward. The exposed functions and attributes as
described in Section \ref{sec:taillard} should be the standard for every \jss
environment. The third improvement would be a standard in the API exposed by
the model. It should be intuitive to provide a policy, environment and
hyperparameters to an \gls{rl}-based scheduling model.

In addition to improving standards to improve comparative studies on \gls{rl}-based
schedulers, future work should be done on reducing retraining time. Comparing
methods for reducing retraining time is an important subject and should be
better executed in the future. If new standards are adopted in the field, it
should cost less time to compare methods for reducing
retraining time. New methods could then be actively compared with other
methods in a controlled, unbiased environment. Cloud providers and future
researchers can then select a method to reduce retraining time in their cloud
environments. Reducing retraining time reduces the energy usage and costs
required for training machine learning algorithms. Reducing the retraining
time is better for cloud providers, end-users and the environment.
