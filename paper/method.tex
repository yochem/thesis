\chapter{Method}

% TODO: multiple models -> specify how many TODO: needs review + adding more

In this chapter the methodology of this thesis is described. In this thesis
approaches to reduce retraining time of \rl based schedulers are compared. The
approaches are compared against multiple \rl based schedulers. This is done to
see how well the approaches work on different \rl algorithms. Firstly the
software and hardware for the experiments is described (§\ref{sec:setup}).
Secondly the data and its usage is described in Section \ref{sec:data}.

% TODO: komt dit erna?

To get the results needed for answering the research questions we first
compare performance of the \rl based schedulers in question
(§\ref{sec:models}), then we integrate the approaches for reducing retraining
time or robustness (§\ref{sec:integrating-robustness}) and at last we evaluate
the robustness and duration of retraining time of these approaches on all \rl
algorithms (§\ref{sec:eval}).


\section{Hardware and Software Specification}\label{sec:setup}

This section explains the hardware and software setup and configuration used
for the research. First the hardware used is described and after this some
important software and its usage is described. The experiments were run on
workstation 8 provided by the Robotics Lab at the University of Amsterdam.
Workstation 8 is an Alienware Aurora R9 1.0.4. The CPU is an Intel i9-9900 (16
threads) running at 5.000 GHz and the GPU is a NVidia 2080 Ti (11 GB) with
CUDA 10 and driver 415. The operating system is Ubuntu 18.04.4. All algorithms
were implemented in the Python programming language\footlink{python.org}. To
prevent versioning issues between dependencies, a virtual Python environment
was created for every algorithm. This also allowed us to run a different
Python version per algorithm. The tool used for creating and managing virtual
Python environments is called \code{pyenv}\footlink{github.com/pyenv/pyenv}.
An example of setting up a virtual environment:

\begin{verbatim}
$ pyenv install 3.7.10
$ pyenv virtualenv 3.7.10 meta-rl
$ pyenv activate
\end{verbatim}

The above commands install Python 3.7.10, create a virtual environment called
`meta-rl' that uses Python version 3.7.10 and activates this virtual
environment. From then on, packages could be installed using the Python
package installer \code{pip} and can only be used in the `meta-rl'
environment. Standard practice is to include a \code{requirements.txt} file in
the root directory of your project. However, only some of the projects used in
this thesis included it. Installing the needed packages at the right version
is as simple as running the command \code{pip install -r requirements.txt}.

Some packages are almost always used when writing \ai software in Python.
These packages are NumPy, Scipy, Keras, Tensorflow or PyTorch\footnotemark.
NumPy and SciPy provide an API for scientific computing and most importantly a
solid multidimensional array object. The other three provide a back-end for
creating well optimized \ml algorithms, especially for neural networks.
Another software project used extensively with this thesis is the OpenAI Gym
\cite{gym2016}. Gym is a for developing and comparing reinforcement learning
algorithms. It mostly provides access to many simple environments to develop
your \rl algorithm on, or as their slogan says ``We provide the environment;
you provide the algorithm.''\footlink{gym.openai.com}.

\footnotetext{Their respective home pages: \link{numpy.org}, \link{scipy.org},
\link{keras.io}, \link{tensorflow.org} and \link{pytorch.org}.}


\section{Data}\label{sec:data}

Four datasets were provided by the supervisors at the start of the thesis. The
datasets described transfered messages in intervals of one or ten minutes. The
four datasets provided 43,200, 182,283, 4,320 and 28,312 rows of data, which
sums up to 258,118 rows of data in total. The datasets all had five columns:
the start time, the transfer time, request number, byte count and completion
status. An overview of the datasets is provided in Table \ref{table:data}.

\begin{table}[htp!]
    \centering
    \caption{Description of the dataset \code{mts\_june\_10m\_rh.csv}. The
    description is somewhat applicable to all four datasets.}
    \label{table:data}
    \begin{tabular}{l|lllll}
        & Start  Time & Transfer Time & Request ID & Bytes & Completed\\ \hline
        Date Type & date & integer  & integer & integer & boolean\\
        Min & - & 145 & - & 8164772 &  -\\
        Max & - & 299061 & - & 165483000000 &  -\\
        Mean & - & 2451 & - & 986461130 &  -\\
    \end{tabular}
\end{table}


The data was complete, i.e. it had zero missing values. Not having missing
values does not mean the data is clean. The dataset contained 3 rows where the
transfer time, request id and byte count had the value of 0. The 3 rows were
dropped because it lead to problems during the conversion of the data into
another format. Another quirk to take note of is that even though the column
name `request ID' would suggest its values will be IDs and thus unique, this
is not the case. For example, there are 1,576 unique IDs in the 4,320 rows of
of the \code{mts\_june\_10m.csv} file. All counts of unique IDs and rows are
shown in Table \ref{table:files}. The commands for getting this data can be
found in appendix \ref{app:dataset}.

\begin{table}[htp!]
    \centering
    \caption{Overview of the four dataset files. The columns `Rows' and
    `Unique IDs' contain the row and ID count, the `Completed' column lists
    the percentage of completed requests (where the completed column equals
    1).}
    \label{table:files}
    \begin{tabular}{l|rrr}
        File & \multicolumn{1}{l}{Rows} & \multicolumn{1}{l}{Unique IDs} & \multicolumn{1}{l}{Completed} \\ \hline
        \code{mts\_june\_10m.csv} & 4320 & 1576 & 44\%\\
        \code{mts\_june\_10m\_rh.csv} & 28311 & 1016 & 87\%\\
        \code{mts\_june\_1m.csv} & 43200 & 655 & 79\%\\
        \code{mts\_june\_1m\_rh.csv} & 182283 & 620 & 94\%\\ \hline
        Total & 258118 & 3867 & 90\%
    \end{tabular}
\end{table}



The provided dataset was not compatible with most models. This is first of all
because it is not a \jss dataset. The data are incoming requests from many
Internet of Things-like devices. The data is sequential and are not different
jobs that need to be scheduled. It is possible to convert the original data to
formats that are more used in \jss tasks. The first format in which the data
was converted was a \DAG. Following the example of the provided data by the
model that uses \DAG data\footnotemark, our provided sequential data was
converted to a \DAG using DAGGEN\footlink{github.com/frs69wq/daggen}. DAGGEN
is a command line tool that generates a \DAG usable by scheduling algorithms.
It has a number of parameters; \code{n} ($n$) is the number of nodes,
\code{mindata} ($D_{min}$) is the minimum size of processed data,
\code{maxdata} ($D_{max}$) is the maximum size of processed data,
\code{density} ($d$) determines the numbers of dependencies between tasks,
\code{ccr} ($ccr$) the communication to computation ratio and \code{fat}
($fat$) the width of the \DAG, i.e. the maximum number of tasks that can be
executed concurrently. Because our provided data was originally not \jss data,
the converted data used parameters and other aspects from the example data. In
the example data, all $n$ values were in the range of 1 to 50. The $n$ values
were determined by our byte count value, which was scaled to be in $[1, 50]$.
Ten outliers were removed to get the mean $n$ closer to the example mean $n$.
The $D_{min}$ and $D_{max}$ were generated from the transfer time values. The
other three parameters, $d$, $ccr$, and $fat$, were randomly chosen. The
distribution of these random values followed the distribution of the parameter
values found in the example data. For $d$ this was an uniform distribution
$[4, 10]$, for $ccr$ it was an uniform distribution in $[3, 6]$ and for $fat$
it was an uniform distribution in $[4, 9]$. There was one other parameter, the
regularity, which was always $0.5$ in the example graphs. An example of a
generated DAGGEN command is the following: \code{daggen -{}-dot -n 4
-{}-mindata 2656 -{}-maxdata 2658 -{}-density 0.4 -{}-ccr 0.5 -{}-fat 0.6
-{}-regular 0.5 -o graphs/4-2657.gv}. The generated graph file from this
command is shown in Fig. \ref{fig:dag}. It can be seen in Fig. \ref{fig:dag}
that the tasks on the lines without an arrow (\code{->}) contain an attribute
\code{expect\_size}. This can not be added with the DAGGEN tool and is
probably added by the authors of the model that uses this data. We noticed
that the expected size is 2.00, 2.50, or 3.33 times smaller than the size
attribute of that task. The distribution was almost uniform: 16500 times a
factor of 2, 16820 times a factor of 2.50, and 16680 times a factor of 3.33.
The expected size was thereafter integrated using the factors as mentioned
above.

\footnotetext{The model called metaRL-offloading uses DAG data. The example
data can be found here: \urlmetadata.}

\begin{figure}[htp!]
    \centering
    \caption{A relatively small graph generated with DAGGEN. The graphs varied
    from 1 line to 381 lines. Each line represents a task. The task have the
    size, alpha and expect\_size attributes.}
    \label{fig:dag}
    \vspace{-.5cm}
    \begin{Verbatim}[frame=single]
// DAG automatically generated by daggen at Thu Jun 10 17:00 2021
// daggen --dot -n 4 --mindata 2656 --maxdata 2658 --density 0.4
//        --ccr 0.5 --fat 0.6 --regular 0.5 -o graphs/4-2657.gv
digraph G {
  1 [size="8589934592", alpha="0.12", expect_size="3435973836"]
  1 -> 3 [size ="33554432"]
  1 -> 4 [size ="33554432"]
  2 [size="42466422292", alpha="0.15", expect_size="16986568916"]
  3 [size="31452222268", alpha="0.15", expect_size="9530976444"]
  4 [size="8589934592", alpha="0.10", expect_size="4294967296"]
}
    \end{Verbatim}
\end{figure}


Other models worked with the popular OpenAI Gym environment. The Gym
environment allows people to build their own environment to make it possible
to test \rl algorithms in other kinds of environments. Gym itself mostly
provides game-like environments. The interface for Gym is very minimal, as
typical \rl algorithms have simple interaction with the environment. Most
importantly, it exposes the following methods and attributes:
\code{env.reset()} which resets the environment and returns the start state,
\code{env.observation\_space} which resembles the current state,
\code{env.action\_space} which are all possible actions in the current state,
and \code{env.step(action)} which performs an action and returns the new
state, a reward for the action and if the model is done.

In \citeA{tassel2021} a gym environment for simulating the job shop scheduling
problem is proposed. This environment provides the bridge between instances
specified in the Taillard specification and the Gym interface. The Taillard
specification, proposed in \citeA{taillard1993}, is a specification for
computing instances. It specifies the number of machines $\mathcal{M}$ and
jobs $\mathcal{J}$ and processing time per job for all machines. An example
for an instance with one job and three machines. The machines are numbered
from 0 to 2 and the processing time of the job is respectively 5, 8 and 9. The
visiting order is 1, 2, 0. The representation of this instance is shown in
Fig. \ref{fig:taillard}.

\begin{figure}[htp!]
    \centering
    \caption{A computing instance following the Taillard specification.}
    \label{fig:taillard}
    \vspace{-.5cm}
    \begin{Verbatim}[frame=single]
1 3
1 8 2 9 0 5
    \end{Verbatim}
\end{figure}


As can be seen, first the number of jobs and the number of machines and then
one job per line, with the visiting order and their processing time. To apply
the provided sequential data to this format, we first analyzed the existing
instances. In the instances provided by \citeA{tassel2021} there were either
40 or 45 machines, evenly distributed. The distribution of the number of jobs
is shown in Fig. \ref{fig:jobs}. Our data was sequential, as is every line in
the Taillard specification. One line describes the path one job travels
between machines. To convert the provided data into we only used the transfer
time. This was converted in the processing time in the Taillard specification.
Then the data was divided in groups of size $\mathcal{M}$, then popped
$\mathcal{J}$ items from these groups, shuffled the machine IDs ($id \in [0,
\mathcal{M})$) and zipped the shuffled machine IDs and the transfer times
together into one line for the Taillard instance.

\begin{figure}[htp!]
    \centering
    %% Creator: Matplotlib, PGF backend
%%
%% To include the figure in your LaTeX document, write
%%   \input{<filename>.pgf}
%%
%% Make sure the required packages are loaded in your preamble
%%   \usepackage{pgf}
%%
%% Figures using additional raster images can only be included by \input if
%% they are in the same directory as the main LaTeX file. For loading figures
%% from other directories you can use the `import` package
%%   \usepackage{import}
%%
%% and then include the figures with
%%   \import{<path to file>}{<filename>.pgf}
%%
%% Matplotlib used the following preamble
%%
\begingroup%
\makeatletter%
\begin{pgfpicture}%
\pgfpathrectangle{\pgfpointorigin}{\pgfqpoint{3.870124in}{2.638191in}}%
\pgfusepath{use as bounding box, clip}%
\begin{pgfscope}%
\pgfsetbuttcap%
\pgfsetmiterjoin%
\pgfsetlinewidth{0.000000pt}%
\definecolor{currentstroke}{rgb}{1.000000,1.000000,1.000000}%
\pgfsetstrokecolor{currentstroke}%
\pgfsetstrokeopacity{0.000000}%
\pgfsetdash{}{0pt}%
\pgfpathmoveto{\pgfqpoint{0.000000in}{0.000000in}}%
\pgfpathlineto{\pgfqpoint{3.870124in}{0.000000in}}%
\pgfpathlineto{\pgfqpoint{3.870124in}{2.638191in}}%
\pgfpathlineto{\pgfqpoint{0.000000in}{2.638191in}}%
\pgfpathclose%
\pgfusepath{}%
\end{pgfscope}%
\begin{pgfscope}%
\pgfsetbuttcap%
\pgfsetmiterjoin%
\definecolor{currentfill}{rgb}{1.000000,1.000000,1.000000}%
\pgfsetfillcolor{currentfill}%
\pgfsetlinewidth{0.000000pt}%
\definecolor{currentstroke}{rgb}{0.000000,0.000000,0.000000}%
\pgfsetstrokecolor{currentstroke}%
\pgfsetstrokeopacity{0.000000}%
\pgfsetdash{}{0pt}%
\pgfpathmoveto{\pgfqpoint{0.515124in}{0.499691in}}%
\pgfpathlineto{\pgfqpoint{3.770124in}{0.499691in}}%
\pgfpathlineto{\pgfqpoint{3.770124in}{2.538191in}}%
\pgfpathlineto{\pgfqpoint{0.515124in}{2.538191in}}%
\pgfpathclose%
\pgfusepath{fill}%
\end{pgfscope}%
\begin{pgfscope}%
\pgfpathrectangle{\pgfqpoint{0.515124in}{0.499691in}}{\pgfqpoint{3.255000in}{2.038500in}}%
\pgfusepath{clip}%
\pgfsetbuttcap%
\pgfsetmiterjoin%
\definecolor{currentfill}{rgb}{0.121569,0.466667,0.705882}%
\pgfsetfillcolor{currentfill}%
\pgfsetlinewidth{0.000000pt}%
\definecolor{currentstroke}{rgb}{0.000000,0.000000,0.000000}%
\pgfsetstrokecolor{currentstroke}%
\pgfsetstrokeopacity{0.000000}%
\pgfsetdash{}{0pt}%
\pgfpathmoveto{\pgfqpoint{0.663078in}{0.499691in}}%
\pgfpathlineto{\pgfqpoint{1.156260in}{0.499691in}}%
\pgfpathlineto{\pgfqpoint{1.156260in}{1.276263in}}%
\pgfpathlineto{\pgfqpoint{0.663078in}{1.276263in}}%
\pgfpathclose%
\pgfusepath{fill}%
\end{pgfscope}%
\begin{pgfscope}%
\pgfpathrectangle{\pgfqpoint{0.515124in}{0.499691in}}{\pgfqpoint{3.255000in}{2.038500in}}%
\pgfusepath{clip}%
\pgfsetbuttcap%
\pgfsetmiterjoin%
\definecolor{currentfill}{rgb}{0.121569,0.466667,0.705882}%
\pgfsetfillcolor{currentfill}%
\pgfsetlinewidth{0.000000pt}%
\definecolor{currentstroke}{rgb}{0.000000,0.000000,0.000000}%
\pgfsetstrokecolor{currentstroke}%
\pgfsetstrokeopacity{0.000000}%
\pgfsetdash{}{0pt}%
\pgfpathmoveto{\pgfqpoint{1.279556in}{0.499691in}}%
\pgfpathlineto{\pgfqpoint{1.772737in}{0.499691in}}%
\pgfpathlineto{\pgfqpoint{1.772737in}{2.130491in}}%
\pgfpathlineto{\pgfqpoint{1.279556in}{2.130491in}}%
\pgfpathclose%
\pgfusepath{fill}%
\end{pgfscope}%
\begin{pgfscope}%
\pgfpathrectangle{\pgfqpoint{0.515124in}{0.499691in}}{\pgfqpoint{3.255000in}{2.038500in}}%
\pgfusepath{clip}%
\pgfsetbuttcap%
\pgfsetmiterjoin%
\definecolor{currentfill}{rgb}{0.121569,0.466667,0.705882}%
\pgfsetfillcolor{currentfill}%
\pgfsetlinewidth{0.000000pt}%
\definecolor{currentstroke}{rgb}{0.000000,0.000000,0.000000}%
\pgfsetstrokecolor{currentstroke}%
\pgfsetstrokeopacity{0.000000}%
\pgfsetdash{}{0pt}%
\pgfpathmoveto{\pgfqpoint{1.896033in}{0.499691in}}%
\pgfpathlineto{\pgfqpoint{2.389215in}{0.499691in}}%
\pgfpathlineto{\pgfqpoint{2.389215in}{2.441120in}}%
\pgfpathlineto{\pgfqpoint{1.896033in}{2.441120in}}%
\pgfpathclose%
\pgfusepath{fill}%
\end{pgfscope}%
\begin{pgfscope}%
\pgfpathrectangle{\pgfqpoint{0.515124in}{0.499691in}}{\pgfqpoint{3.255000in}{2.038500in}}%
\pgfusepath{clip}%
\pgfsetbuttcap%
\pgfsetmiterjoin%
\definecolor{currentfill}{rgb}{0.121569,0.466667,0.705882}%
\pgfsetfillcolor{currentfill}%
\pgfsetlinewidth{0.000000pt}%
\definecolor{currentstroke}{rgb}{0.000000,0.000000,0.000000}%
\pgfsetstrokecolor{currentstroke}%
\pgfsetstrokeopacity{0.000000}%
\pgfsetdash{}{0pt}%
\pgfpathmoveto{\pgfqpoint{2.512510in}{0.499691in}}%
\pgfpathlineto{\pgfqpoint{3.005692in}{0.499691in}}%
\pgfpathlineto{\pgfqpoint{3.005692in}{2.052834in}}%
\pgfpathlineto{\pgfqpoint{2.512510in}{2.052834in}}%
\pgfpathclose%
\pgfusepath{fill}%
\end{pgfscope}%
\begin{pgfscope}%
\pgfpathrectangle{\pgfqpoint{0.515124in}{0.499691in}}{\pgfqpoint{3.255000in}{2.038500in}}%
\pgfusepath{clip}%
\pgfsetbuttcap%
\pgfsetmiterjoin%
\definecolor{currentfill}{rgb}{0.121569,0.466667,0.705882}%
\pgfsetfillcolor{currentfill}%
\pgfsetlinewidth{0.000000pt}%
\definecolor{currentstroke}{rgb}{0.000000,0.000000,0.000000}%
\pgfsetstrokecolor{currentstroke}%
\pgfsetstrokeopacity{0.000000}%
\pgfsetdash{}{0pt}%
\pgfpathmoveto{\pgfqpoint{3.128987in}{0.499691in}}%
\pgfpathlineto{\pgfqpoint{3.622169in}{0.499691in}}%
\pgfpathlineto{\pgfqpoint{3.622169in}{1.276263in}}%
\pgfpathlineto{\pgfqpoint{3.128987in}{1.276263in}}%
\pgfpathclose%
\pgfusepath{fill}%
\end{pgfscope}%
\begin{pgfscope}%
\pgfsetbuttcap%
\pgfsetroundjoin%
\definecolor{currentfill}{rgb}{0.000000,0.000000,0.000000}%
\pgfsetfillcolor{currentfill}%
\pgfsetlinewidth{0.803000pt}%
\definecolor{currentstroke}{rgb}{0.000000,0.000000,0.000000}%
\pgfsetstrokecolor{currentstroke}%
\pgfsetdash{}{0pt}%
\pgfsys@defobject{currentmarker}{\pgfqpoint{0.000000in}{-0.048611in}}{\pgfqpoint{0.000000in}{0.000000in}}{%
\pgfpathmoveto{\pgfqpoint{0.000000in}{0.000000in}}%
\pgfpathlineto{\pgfqpoint{0.000000in}{-0.048611in}}%
\pgfusepath{stroke,fill}%
}%
\begin{pgfscope}%
\pgfsys@transformshift{0.909669in}{0.499691in}%
\pgfsys@useobject{currentmarker}{}%
\end{pgfscope}%
\end{pgfscope}%
\begin{pgfscope}%
\definecolor{textcolor}{rgb}{0.000000,0.000000,0.000000}%
\pgfsetstrokecolor{textcolor}%
\pgfsetfillcolor{textcolor}%
\pgftext[x=0.909669in,y=0.402469in,,top]{\color{textcolor}\rmfamily\fontsize{10.000000}{12.000000}\selectfont 15}%
\end{pgfscope}%
\begin{pgfscope}%
\pgfsetbuttcap%
\pgfsetroundjoin%
\definecolor{currentfill}{rgb}{0.000000,0.000000,0.000000}%
\pgfsetfillcolor{currentfill}%
\pgfsetlinewidth{0.803000pt}%
\definecolor{currentstroke}{rgb}{0.000000,0.000000,0.000000}%
\pgfsetstrokecolor{currentstroke}%
\pgfsetdash{}{0pt}%
\pgfsys@defobject{currentmarker}{\pgfqpoint{0.000000in}{-0.048611in}}{\pgfqpoint{0.000000in}{0.000000in}}{%
\pgfpathmoveto{\pgfqpoint{0.000000in}{0.000000in}}%
\pgfpathlineto{\pgfqpoint{0.000000in}{-0.048611in}}%
\pgfusepath{stroke,fill}%
}%
\begin{pgfscope}%
\pgfsys@transformshift{1.526146in}{0.499691in}%
\pgfsys@useobject{currentmarker}{}%
\end{pgfscope}%
\end{pgfscope}%
\begin{pgfscope}%
\definecolor{textcolor}{rgb}{0.000000,0.000000,0.000000}%
\pgfsetstrokecolor{textcolor}%
\pgfsetfillcolor{textcolor}%
\pgftext[x=1.526146in,y=0.402469in,,top]{\color{textcolor}\rmfamily\fontsize{10.000000}{12.000000}\selectfont 20}%
\end{pgfscope}%
\begin{pgfscope}%
\pgfsetbuttcap%
\pgfsetroundjoin%
\definecolor{currentfill}{rgb}{0.000000,0.000000,0.000000}%
\pgfsetfillcolor{currentfill}%
\pgfsetlinewidth{0.803000pt}%
\definecolor{currentstroke}{rgb}{0.000000,0.000000,0.000000}%
\pgfsetstrokecolor{currentstroke}%
\pgfsetdash{}{0pt}%
\pgfsys@defobject{currentmarker}{\pgfqpoint{0.000000in}{-0.048611in}}{\pgfqpoint{0.000000in}{0.000000in}}{%
\pgfpathmoveto{\pgfqpoint{0.000000in}{0.000000in}}%
\pgfpathlineto{\pgfqpoint{0.000000in}{-0.048611in}}%
\pgfusepath{stroke,fill}%
}%
\begin{pgfscope}%
\pgfsys@transformshift{2.142624in}{0.499691in}%
\pgfsys@useobject{currentmarker}{}%
\end{pgfscope}%
\end{pgfscope}%
\begin{pgfscope}%
\definecolor{textcolor}{rgb}{0.000000,0.000000,0.000000}%
\pgfsetstrokecolor{textcolor}%
\pgfsetfillcolor{textcolor}%
\pgftext[x=2.142624in,y=0.402469in,,top]{\color{textcolor}\rmfamily\fontsize{10.000000}{12.000000}\selectfont 30}%
\end{pgfscope}%
\begin{pgfscope}%
\pgfsetbuttcap%
\pgfsetroundjoin%
\definecolor{currentfill}{rgb}{0.000000,0.000000,0.000000}%
\pgfsetfillcolor{currentfill}%
\pgfsetlinewidth{0.803000pt}%
\definecolor{currentstroke}{rgb}{0.000000,0.000000,0.000000}%
\pgfsetstrokecolor{currentstroke}%
\pgfsetdash{}{0pt}%
\pgfsys@defobject{currentmarker}{\pgfqpoint{0.000000in}{-0.048611in}}{\pgfqpoint{0.000000in}{0.000000in}}{%
\pgfpathmoveto{\pgfqpoint{0.000000in}{0.000000in}}%
\pgfpathlineto{\pgfqpoint{0.000000in}{-0.048611in}}%
\pgfusepath{stroke,fill}%
}%
\begin{pgfscope}%
\pgfsys@transformshift{2.759101in}{0.499691in}%
\pgfsys@useobject{currentmarker}{}%
\end{pgfscope}%
\end{pgfscope}%
\begin{pgfscope}%
\definecolor{textcolor}{rgb}{0.000000,0.000000,0.000000}%
\pgfsetstrokecolor{textcolor}%
\pgfsetfillcolor{textcolor}%
\pgftext[x=2.759101in,y=0.402469in,,top]{\color{textcolor}\rmfamily\fontsize{10.000000}{12.000000}\selectfont 50}%
\end{pgfscope}%
\begin{pgfscope}%
\pgfsetbuttcap%
\pgfsetroundjoin%
\definecolor{currentfill}{rgb}{0.000000,0.000000,0.000000}%
\pgfsetfillcolor{currentfill}%
\pgfsetlinewidth{0.803000pt}%
\definecolor{currentstroke}{rgb}{0.000000,0.000000,0.000000}%
\pgfsetstrokecolor{currentstroke}%
\pgfsetdash{}{0pt}%
\pgfsys@defobject{currentmarker}{\pgfqpoint{0.000000in}{-0.048611in}}{\pgfqpoint{0.000000in}{0.000000in}}{%
\pgfpathmoveto{\pgfqpoint{0.000000in}{0.000000in}}%
\pgfpathlineto{\pgfqpoint{0.000000in}{-0.048611in}}%
\pgfusepath{stroke,fill}%
}%
\begin{pgfscope}%
\pgfsys@transformshift{3.375578in}{0.499691in}%
\pgfsys@useobject{currentmarker}{}%
\end{pgfscope}%
\end{pgfscope}%
\begin{pgfscope}%
\definecolor{textcolor}{rgb}{0.000000,0.000000,0.000000}%
\pgfsetstrokecolor{textcolor}%
\pgfsetfillcolor{textcolor}%
\pgftext[x=3.375578in,y=0.402469in,,top]{\color{textcolor}\rmfamily\fontsize{10.000000}{12.000000}\selectfont 100}%
\end{pgfscope}%
\begin{pgfscope}%
\definecolor{textcolor}{rgb}{0.000000,0.000000,0.000000}%
\pgfsetstrokecolor{textcolor}%
\pgfsetfillcolor{textcolor}%
\pgftext[x=2.142624in,y=0.223457in,,top]{\color{textcolor}\rmfamily\fontsize{10.000000}{12.000000}\selectfont Number of jobs}%
\end{pgfscope}%
\begin{pgfscope}%
\pgfsetbuttcap%
\pgfsetroundjoin%
\definecolor{currentfill}{rgb}{0.000000,0.000000,0.000000}%
\pgfsetfillcolor{currentfill}%
\pgfsetlinewidth{0.803000pt}%
\definecolor{currentstroke}{rgb}{0.000000,0.000000,0.000000}%
\pgfsetstrokecolor{currentstroke}%
\pgfsetdash{}{0pt}%
\pgfsys@defobject{currentmarker}{\pgfqpoint{-0.048611in}{0.000000in}}{\pgfqpoint{-0.000000in}{0.000000in}}{%
\pgfpathmoveto{\pgfqpoint{-0.000000in}{0.000000in}}%
\pgfpathlineto{\pgfqpoint{-0.048611in}{0.000000in}}%
\pgfusepath{stroke,fill}%
}%
\begin{pgfscope}%
\pgfsys@transformshift{0.515124in}{0.499691in}%
\pgfsys@useobject{currentmarker}{}%
\end{pgfscope}%
\end{pgfscope}%
\begin{pgfscope}%
\definecolor{textcolor}{rgb}{0.000000,0.000000,0.000000}%
\pgfsetstrokecolor{textcolor}%
\pgfsetfillcolor{textcolor}%
\pgftext[x=0.348457in, y=0.451466in, left, base]{\color{textcolor}\rmfamily\fontsize{10.000000}{12.000000}\selectfont \(\displaystyle {0}\)}%
\end{pgfscope}%
\begin{pgfscope}%
\pgfsetbuttcap%
\pgfsetroundjoin%
\definecolor{currentfill}{rgb}{0.000000,0.000000,0.000000}%
\pgfsetfillcolor{currentfill}%
\pgfsetlinewidth{0.803000pt}%
\definecolor{currentstroke}{rgb}{0.000000,0.000000,0.000000}%
\pgfsetstrokecolor{currentstroke}%
\pgfsetdash{}{0pt}%
\pgfsys@defobject{currentmarker}{\pgfqpoint{-0.048611in}{0.000000in}}{\pgfqpoint{-0.000000in}{0.000000in}}{%
\pgfpathmoveto{\pgfqpoint{-0.000000in}{0.000000in}}%
\pgfpathlineto{\pgfqpoint{-0.048611in}{0.000000in}}%
\pgfusepath{stroke,fill}%
}%
\begin{pgfscope}%
\pgfsys@transformshift{0.515124in}{0.887977in}%
\pgfsys@useobject{currentmarker}{}%
\end{pgfscope}%
\end{pgfscope}%
\begin{pgfscope}%
\definecolor{textcolor}{rgb}{0.000000,0.000000,0.000000}%
\pgfsetstrokecolor{textcolor}%
\pgfsetfillcolor{textcolor}%
\pgftext[x=0.348457in, y=0.839752in, left, base]{\color{textcolor}\rmfamily\fontsize{10.000000}{12.000000}\selectfont \(\displaystyle {5}\)}%
\end{pgfscope}%
\begin{pgfscope}%
\pgfsetbuttcap%
\pgfsetroundjoin%
\definecolor{currentfill}{rgb}{0.000000,0.000000,0.000000}%
\pgfsetfillcolor{currentfill}%
\pgfsetlinewidth{0.803000pt}%
\definecolor{currentstroke}{rgb}{0.000000,0.000000,0.000000}%
\pgfsetstrokecolor{currentstroke}%
\pgfsetdash{}{0pt}%
\pgfsys@defobject{currentmarker}{\pgfqpoint{-0.048611in}{0.000000in}}{\pgfqpoint{-0.000000in}{0.000000in}}{%
\pgfpathmoveto{\pgfqpoint{-0.000000in}{0.000000in}}%
\pgfpathlineto{\pgfqpoint{-0.048611in}{0.000000in}}%
\pgfusepath{stroke,fill}%
}%
\begin{pgfscope}%
\pgfsys@transformshift{0.515124in}{1.276263in}%
\pgfsys@useobject{currentmarker}{}%
\end{pgfscope}%
\end{pgfscope}%
\begin{pgfscope}%
\definecolor{textcolor}{rgb}{0.000000,0.000000,0.000000}%
\pgfsetstrokecolor{textcolor}%
\pgfsetfillcolor{textcolor}%
\pgftext[x=0.279012in, y=1.228037in, left, base]{\color{textcolor}\rmfamily\fontsize{10.000000}{12.000000}\selectfont \(\displaystyle {10}\)}%
\end{pgfscope}%
\begin{pgfscope}%
\pgfsetbuttcap%
\pgfsetroundjoin%
\definecolor{currentfill}{rgb}{0.000000,0.000000,0.000000}%
\pgfsetfillcolor{currentfill}%
\pgfsetlinewidth{0.803000pt}%
\definecolor{currentstroke}{rgb}{0.000000,0.000000,0.000000}%
\pgfsetstrokecolor{currentstroke}%
\pgfsetdash{}{0pt}%
\pgfsys@defobject{currentmarker}{\pgfqpoint{-0.048611in}{0.000000in}}{\pgfqpoint{-0.000000in}{0.000000in}}{%
\pgfpathmoveto{\pgfqpoint{-0.000000in}{0.000000in}}%
\pgfpathlineto{\pgfqpoint{-0.048611in}{0.000000in}}%
\pgfusepath{stroke,fill}%
}%
\begin{pgfscope}%
\pgfsys@transformshift{0.515124in}{1.664548in}%
\pgfsys@useobject{currentmarker}{}%
\end{pgfscope}%
\end{pgfscope}%
\begin{pgfscope}%
\definecolor{textcolor}{rgb}{0.000000,0.000000,0.000000}%
\pgfsetstrokecolor{textcolor}%
\pgfsetfillcolor{textcolor}%
\pgftext[x=0.279012in, y=1.616323in, left, base]{\color{textcolor}\rmfamily\fontsize{10.000000}{12.000000}\selectfont \(\displaystyle {15}\)}%
\end{pgfscope}%
\begin{pgfscope}%
\pgfsetbuttcap%
\pgfsetroundjoin%
\definecolor{currentfill}{rgb}{0.000000,0.000000,0.000000}%
\pgfsetfillcolor{currentfill}%
\pgfsetlinewidth{0.803000pt}%
\definecolor{currentstroke}{rgb}{0.000000,0.000000,0.000000}%
\pgfsetstrokecolor{currentstroke}%
\pgfsetdash{}{0pt}%
\pgfsys@defobject{currentmarker}{\pgfqpoint{-0.048611in}{0.000000in}}{\pgfqpoint{-0.000000in}{0.000000in}}{%
\pgfpathmoveto{\pgfqpoint{-0.000000in}{0.000000in}}%
\pgfpathlineto{\pgfqpoint{-0.048611in}{0.000000in}}%
\pgfusepath{stroke,fill}%
}%
\begin{pgfscope}%
\pgfsys@transformshift{0.515124in}{2.052834in}%
\pgfsys@useobject{currentmarker}{}%
\end{pgfscope}%
\end{pgfscope}%
\begin{pgfscope}%
\definecolor{textcolor}{rgb}{0.000000,0.000000,0.000000}%
\pgfsetstrokecolor{textcolor}%
\pgfsetfillcolor{textcolor}%
\pgftext[x=0.279012in, y=2.004609in, left, base]{\color{textcolor}\rmfamily\fontsize{10.000000}{12.000000}\selectfont \(\displaystyle {20}\)}%
\end{pgfscope}%
\begin{pgfscope}%
\pgfsetbuttcap%
\pgfsetroundjoin%
\definecolor{currentfill}{rgb}{0.000000,0.000000,0.000000}%
\pgfsetfillcolor{currentfill}%
\pgfsetlinewidth{0.803000pt}%
\definecolor{currentstroke}{rgb}{0.000000,0.000000,0.000000}%
\pgfsetstrokecolor{currentstroke}%
\pgfsetdash{}{0pt}%
\pgfsys@defobject{currentmarker}{\pgfqpoint{-0.048611in}{0.000000in}}{\pgfqpoint{-0.000000in}{0.000000in}}{%
\pgfpathmoveto{\pgfqpoint{-0.000000in}{0.000000in}}%
\pgfpathlineto{\pgfqpoint{-0.048611in}{0.000000in}}%
\pgfusepath{stroke,fill}%
}%
\begin{pgfscope}%
\pgfsys@transformshift{0.515124in}{2.441120in}%
\pgfsys@useobject{currentmarker}{}%
\end{pgfscope}%
\end{pgfscope}%
\begin{pgfscope}%
\definecolor{textcolor}{rgb}{0.000000,0.000000,0.000000}%
\pgfsetstrokecolor{textcolor}%
\pgfsetfillcolor{textcolor}%
\pgftext[x=0.279012in, y=2.392894in, left, base]{\color{textcolor}\rmfamily\fontsize{10.000000}{12.000000}\selectfont \(\displaystyle {25}\)}%
\end{pgfscope}%
\begin{pgfscope}%
\definecolor{textcolor}{rgb}{0.000000,0.000000,0.000000}%
\pgfsetstrokecolor{textcolor}%
\pgfsetfillcolor{textcolor}%
\pgftext[x=0.223457in,y=1.518941in,,bottom,rotate=90.000000]{\color{textcolor}\rmfamily\fontsize{10.000000}{12.000000}\selectfont Count in original data}%
\end{pgfscope}%
\begin{pgfscope}%
\pgfsetrectcap%
\pgfsetmiterjoin%
\pgfsetlinewidth{0.803000pt}%
\definecolor{currentstroke}{rgb}{0.000000,0.000000,0.000000}%
\pgfsetstrokecolor{currentstroke}%
\pgfsetdash{}{0pt}%
\pgfpathmoveto{\pgfqpoint{0.515124in}{0.499691in}}%
\pgfpathlineto{\pgfqpoint{0.515124in}{2.538191in}}%
\pgfusepath{stroke}%
\end{pgfscope}%
\begin{pgfscope}%
\pgfsetrectcap%
\pgfsetmiterjoin%
\pgfsetlinewidth{0.803000pt}%
\definecolor{currentstroke}{rgb}{0.000000,0.000000,0.000000}%
\pgfsetstrokecolor{currentstroke}%
\pgfsetdash{}{0pt}%
\pgfpathmoveto{\pgfqpoint{3.770124in}{0.499691in}}%
\pgfpathlineto{\pgfqpoint{3.770124in}{2.538191in}}%
\pgfusepath{stroke}%
\end{pgfscope}%
\begin{pgfscope}%
\pgfsetrectcap%
\pgfsetmiterjoin%
\pgfsetlinewidth{0.803000pt}%
\definecolor{currentstroke}{rgb}{0.000000,0.000000,0.000000}%
\pgfsetstrokecolor{currentstroke}%
\pgfsetdash{}{0pt}%
\pgfpathmoveto{\pgfqpoint{0.515124in}{0.499691in}}%
\pgfpathlineto{\pgfqpoint{3.770124in}{0.499691in}}%
\pgfusepath{stroke}%
\end{pgfscope}%
\begin{pgfscope}%
\pgfsetrectcap%
\pgfsetmiterjoin%
\pgfsetlinewidth{0.803000pt}%
\definecolor{currentstroke}{rgb}{0.000000,0.000000,0.000000}%
\pgfsetstrokecolor{currentstroke}%
\pgfsetdash{}{0pt}%
\pgfpathmoveto{\pgfqpoint{0.515124in}{2.538191in}}%
\pgfpathlineto{\pgfqpoint{3.770124in}{2.538191in}}%
\pgfusepath{stroke}%
\end{pgfscope}%
\end{pgfpicture}%
\makeatother%
\endgroup%

    \caption{Distribution of number of jobs in data from Tassel et al. (2021).}
    \label{fig:jobs}
\end{figure}







% \section{Selection of Models}\label{sec:models}

% \section{Training the Models}\label{sec:train}

% \section{Integrating Approaches for
% Robustness}\label{sec:integrating-robustness}

% \section{Evaluation}\label{sec:eval}
