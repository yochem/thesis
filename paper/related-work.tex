\chapter{Related Work}

In this chapter related work is discussed. In the first section the scope of
the thesis is defined and the methodology of finding related work is
described. Thereafter related work on the topics of job scheduling and its
NP-hardness, \rl based schedulers, assessing robustness of \rl methods and
approaches for reducing retraining time is discussed. Finally, the gap of in
the related work is shown.


\section{Scope and Methodology}

Comparing different methods and models is the main task in this thesis. To
compare methods of models, a justified way of assessing methods and models is
needed. It is also important that the comparison is done on the indicators on
which the performance depends. These are called the \kpis.

\note{unfinished}


\section{Job Scheduling}\label{sec:scheduling}

Imagine cloud systems as described in Section \ref{sec:cloud}. These systems
have many tasks, or as in this thesis called jobs, that need to be executed
for it to function. Jobs are scheduled by a job scheduler or simply called the
scheduler. These jobs vary a lot in duration time and resource-need. To
illustrate, if the cloud environment provides a back-end for a website, a job
could be one of the following: Serving the right HTML page to a visiting user,
compressing user uploaded images, spam filtering, detecting anomalies in user
logins or many other types of jobs that need to be executed to provide a fully
functioning website. The schedulers distributes jobs over many resources,
provided by a cloud environment. Optimally distributing the jobs is important
for cloud environments, because efficiently using resources means there are
less resources needed and jobs will complete in a shorter time. Thus, well
working schedulers are important for cloud environments, because efficiently
using resources saves money and is easier to maintain.

What is efficient and how can schedulers be efficient? There are many metrics
on which optimization can be done to make a scheduler more efficient. For
example minimizing job slowdown time, minimizing average completion time,
maximizing throughput (jobs per time unit) or minimizing total completion time
(the makespan). Schedulers are created to be optimized on one or more of these
aspects, varying the importance. Unfortunately, the optimization of schedulers
is complex. It is a well-known problem in computer science, because of its
NP-hardness. In the next paragraph is explained what it means to be a NP-hard
problem and the reason that job scheduling is NP-hard.


\subsection{NP-Hard}

NP-hardness is a term used in the P versus NP problem. This problem is one of
the seven Millennium Prize Problems \cite{carlson2006} and still unsolved. The
P versus NP problem is about computational complexity, a way of categorizing
problems based on `how hard' they are. Computational problems have two
complexity aspects: the complexity to solve the problem and the complexity to
verify if the solution to the problem is correct. The question is if the
solution to a given problem can be verified quickly (in polynomial time), is
there an algorithm that can find the solution quickly? If this is true, P
equals NP. Many computer scientists believe this is not the case for all
problems \cite{rosenberger2012}. Believed is that there are problems that can
be verified quickly but not solved quickly, which are called NP-hard problems.
A well-known NP-hard problem is the Sudoku puzzle, especially larger ones
\cite{yato2003}.

Job scheduling is a generalized version of the traveling salesperson problem.
The problem is as follows: ``Given a list of cities and the distances between
each pair of cities, what is the shortest possible route that visits each city
exactly once and returns to the origin city?'' \cite{flood1956}. In this
problem, the cities are the resources and the salesman is the job.

Job scheduling is NP-hard because it can be derived from the Graph Coloring
Problem, as done in \citeA{karp1972}. The graph coloring problem is a NP-hard
problem itself.


\section{Reinforcement Learning}\label{sec:rl}

Most \ml based schedulers are \rl based, and therefore this thesis
focuses on reducing retraining of \rl models. The reason most \ml based
schedulers are \rl based is because of how \rl works and differs from the other
\ml paradigms. \rl and its differences with the other paradigms is very well
explained in `the most popular artificial intelligence textbook in the
world'\footnotemark, \citeA{russell2010}. Their explanation of \rl is
summarized in the following paragraph.

\footnotetext{According to
\href{https://www.tor.com/2011/06/21/norvig-vs-chomsky-and-the-fight-for-the-future-of-ai/}{this
blog}, but also shown on the homepage of the book
(\url{http://aima.cs.berkeley.edu/}).}

\rl is one of the basic paradigms in \ml, along with supervised
learning and unsupervised learning. \rl is different from the other two
paradigms. Supervised learning and unsupervised learning have a thing in
common: the need for data. Supervised learning needs annotated data, various
inputs and desired outputs are given and the algorithm learns a function to
get as close to the wanted outputs as possible given the inputs. With
unsupervised learning a model is forced to build an internal representation of
the world by mimicking the data. The reason \rl is different is that it does
not depend on data, but rather learns from a feedback loop of rewards or
reinforcements. It typically consists of one or more agents, an environment, a
set of actions and a set of states. This agent performs actions in the
environment to reach states. An agent receives a reward for taking actions. In
many complex environments \rl is the only feasible way to train models because
there might be little data available or the environment is too complex to
model \cite{russell2010}. For the reason that \rl has no need for input data
but solely needs an environment, actions and a reward function it is widely
used in research about using \ai in resource managers. The popularity of \rl and
the nature of how \rl works is the reason this research is narrowed down to
reducing retraining of \rl schedulers.


\section{Evaluation of Reinforcement Learning Models}
% NOTE: How can we effectively assess and compare RL based methods in job
% scheduling?

Currently there is not yet a standard evaluation method for \rl models. Due to
the absence of a standard evaluation model, reproducibility often becomes more
difficult. Work is done in defining standard evaluation methods, but these
methods are not yet the industry standard. An evaluation method that is
currently used is regret. Regret can only be used as a metric when an agent
with optimal policy can be defined. The difference in actions taken is the
regret per action. This is defined as the reward for the action of the optimal
agent minus the reward for the taken action. Knowing the optimal policy is not
always possible.

There are also non-scientific metrics used to evaluate \rl models. Some
non-scientific metrics that were used to evaluate \rl models are popular,
well-defined environments like the openAI gym \cite{gym2016}. A comparison of
\rl models is done on their
leaderboard\footlink{github.com/openai/gym/wiki/Leaderboard}. The used metric
is the number of episodes it took to solve the problem. More scientific
evaluation methodologies have been proposed, e.g. a framework for comparison
in \citeA{khetarpal2018} and a novel evaluation method in \citeA{jordan2020}.

\section{Assessing Robustness of Reinforcement Learning}


\note{NOTE: What are the indicators to assess robustness of \rl based
schedulers?}

\note{NOTE: What are the state-of-the-art approaches for reducing retraining
time?}

\section{Approaches to Reduce Retraining Time}



\section{Reducing Retraining of Schedulers}
\note{Gap:} In Chapter \ref{sec:intro} multiple \rl based schedulers were
discussed. These schedulers lack robustness: flexible environments are not
handled well. Methods for improving robustness of \rl models exist, but are not
yet integrated in \rl based schedulers. Integrating these methods would improve
robustness and reduce retraining time.
