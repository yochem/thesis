\chapter{Discussion}\label{sec:discussion}

\section{Achievements}

During this research project there were many larger and smaller achievements.
We discuss three achievements in this section.

Firstly, right at the start of the project, the two first chapters (§\ref{sec:intro}
and §\ref{sec:state-of-the-art}) were written. Because the BSc. \ai does not
require much academic writing, writing was not one of the best skills.
However, during writing, this developed. Continuing with the other chapters
after the experiments was easier and costed a lot less time. One of my
achievements therefor is the improvement in English academic writing.

Secondly, during the experiments, a lot if projects with very specific Python
dependencies had to be run on the same machine. Virtual environments were not
necessary for us to be used before this project. After the experiments a
better understanding of Python's virtual environments and the knowledge of the
advantages of containerizing these environments has developed.

Lastly, during the whole project, much is learned about the fields of cloud
computing and \jss. Before the project there was almost no knowledge about
this fields. The only aspect of this thesis with some foreknowledge are the
workings of reinforcement learning and python programming. Due to the lack of
knowledge in cloud computing and scheduling much information had to be read to
get somewhat of an understanding of it. It is a complex field working on very
complex technology. It was very interesting to get some insights in these
fields.


\section{Shortcomings}

In this section the shortcoming of this research are discussed. During the
experiments it was not managed to fulfil all experiments to properly answer
the research question. The reasons for this are discussed in the following
subsections.


\subsection{Papers with Code}

Nowadays, many papers include the corresponding
code\footlink{paperswithcode.com}. Although this is generally a good thing
that should be encouraged, it is not always usable yet. All code repositories
that accompanied related work, lacked documentation. The paper itself does not
document how code works and additional documentation is not written either.
Working with the codebase required reading the written code, that usually also
lacks comments and good design patters, and figuring out how to adjust it to
work with your own data, code, or environment. To give an example, the \mrlco
code\footlink{github.com/linkpark/metarl-offloading} only specifies the entry
point of the code and which packages should be installed. The list of packages
is incomplete and it is not specified that an old version of Python should be
installed. Another example from the same code repository is that the command
for generating a specific graph is commented in the first line of the file.
But when running these commands, the jobs lack an \code{expect\_size}
attribute. This is not something that can be done with \code{DAGGEN} and must
be later on added by the authors of the research paper. This should definitely
be documented. Academic code is also error-prone with unclear error messages as
result. The unclear error messages require lots of time to debug, which is not
doable in a time-critic project like the thesis. To conclude, the shortcoming
was the expectation that code provided with papers would be industry-grade
code with clear documentation. This was not the case with most projects, which
leads into time-consuming debug sessions. Because of this, I had less time for
the actual experiments. Although papers that include code are a good step in
the right direction, it is not very useful yet because of the low quality code
and documentation. Due to lack of documentation it is necessary to read the
code to be able to use it, which can cost much time.


\subsection{Standards in Reinforcement Learning}\label{sec:standards}

At the start of the experiments it was clear that the papers on \rl and
scheduling algorithms lack standards in data format and API. We found no
standard for this during literature review. This made comparing and assessing
them much harder. It was not possible to loop through the models, fit them on
the same data and get the same metrics from them. It also was not possible to
reuse code written for one specific algorithm. Data had to be converted into
data formats that differ very much from each other. From graph structure to a
specification dating from \citeyearNP{taillard1993}. This did not match the
expectations. Many times in the BSc. \ai, data would be presented in a
\textit{Pandas DataFrame} and algorithms would work on this tabular data.
Looking back, the assumption that \jss data would also generally be in tabular
form should not have been made. There should be a standard in the field of \rl
and \jss for data and code APIs to easily swap \rl algorithms and train many
different \rl algorithms on the same data or environment. Good steps have been
made by OpenAI with their Gym environment, but the \jss environment is not
good yet.


\subsection{Generating Data}

A popular method for training \ai schedulers is to generate instances. There
are many advantages: generating random environments is easy and can still be
tweaked by parameters. The data does not have to be saved on disk and there is
no I/O overhead. Random environments are also a good exercise: there are
likely no patterns in incoming jobs. Because of this, the \ai will not just
recognize patterns and change approach accordingly. But generating data has
its disadvantages, e.g. not reproducible due to randomness, unclear how
parameters might affect the generated data, not the right probability
distribution in randomness, and it might differ to much from real world data.
It is also hard to detect possible overfitting on certain data aspects and can
introduce biases \cite{hall2001}.

Another important problem with generated data is that it is hard to compare
performance between two algorithms using different data generators. Even if
the parameters are all equal, it still might have different random
distributions, other logic or some magic numbers in it. The problem with
different data input formats as discussed in Section \ref{sec:standards} also
exists when the data is generated.


\subsection{Concluding Shortcomings}

The results are not the expected results from the start of the research
project. Most of the shortcomings that led to not achieving the intended
results are due to the models that do not meet the expectations on
standardization and clear documentation. Converting the data for every model
that would be evaluated was not feasible in the amount of time for this
project.
